\documentclass{bioinfo}
\copyrightyear{2017} \pubyear{2017}
\usepackage[ngerman]{babel}
\access{24.11.2017}
\appnotes{Bericht}

\begin{document}
\firstpage{1}

\subtitle{Algorithmische Bioinformatik}

\title[short Title]{Praktikum 1}
\author[Sample \textit{et~al}.]{Timur Masar\,$^{\text{\sfb}}$, Celvic Coomber\,$^{\text{\sfb}}$ und Gergana Stanilova\,$^{\text{\sfb}}$}
\address{$^{\text{\sf}}$Mathematik und Informatik, Freie Universit"at Berlin}
\corresp{$^\ast$Tutor: Marc}
\history{}
\editor{}

\abstract{\textbf{Motivation:}  Paper on "How to write a good abstract: https://www.ncbi.nlm.nih.gov/pmc/articles/PMC3136027/"\\
\textbf{Results:} \\
\textbf{Availability:} \\
}

\maketitle

\section{Introduction}


%\enlargethispage{12pt}

\section{Approach and Methods}
\begin{itemize}
\item Java
\item IntelliJ
\item Folien f"ur die 3 Algorithmen
\item NCBI DB, 500er Fenster
\end{itemize}
\begin{methods}
\section{Results}

\begin{itemize}
\item mindestens eine Angabe "uber die am h"aufigsten vorkommenden 3-mere und 9-mere
\item eine graphische Darstellung der gemessenen Laufzeiten (pro Algorithmus) in Abh"angigkeit von der Eingabe-Sequenzl"ange
\item  
\end{itemize}

%\begin{figure}[!tpb]%figure2
%%\centerline{\includegraphics{fig02.eps}}
%\caption{Caption, caption.}\label{fig:02}
%\end{figure}

\end{methods}
\section{Discussion}

\begin{itemize}
\item Mit den Anworten in Rosalind vergleichen
\item Laufzeit der Algorithmen, O-Notation 
\item  
\end{itemize}

\section{Conclusion}
Lassen wir diesen Teil weg?

\section*{Acknowledgements}
Und auch diesen?
%\bibliographystyle{natbib}
%\bibliographystyle{achemnat}
%\bibliographystyle{plainnat}
%\bibliographystyle{abbrv}
%\bibliographystyle{bioinformatics}
%
%\bibliographystyle{plain}
%
%\bibliography{Document}

\begin{thebibliography}{}
Wollen wir die Referenzen mit einer .bib-Datei auflisten?


%\bibitem[Bag {\it et~al}., 2001]{Bag01}
%Bag,M., Name2, Name3 (2001) Article title, {\it Journal Name}, {\bf 99}, 33-54.


\end{thebibliography}
\end{document}

